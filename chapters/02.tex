\chapter{Phonology}

\section{Vowels}
Tuatschin possesses nine vowels which are presented in \tabref{vow}. Minimal pairs are listed in \tabref{vmp}.
\begin{table}
\caption{Vowels}
\label{vow}
 \begin{tabular}{lllll}
  \lsptoprule
            &  front& central  & near back & back \\
  \midrule
 close   &  i  &      &   &    u \\
 near close    &   &      & ʊ  &  \\
close-mid    &  e  &   &  &       \\
mid    & ɛ   & ə  &        & ɔ \\
near open    &    &  ɐ   &        \\
open    &   a  & \\
  \lspbottomrule
 \end{tabular}
\end{table}

The reduced vowels [ə] and [ɐ] only occur in unstressed syllables. There are no minimal pairs contrasting [ə] and [ɐ], and the distribution of these two reduced vowels is not clear to me. It seems as if in certain cases a speaker may use [ə] or [ɐ] in the same environment; there is, however, a tendency for [ɐ] to occur in the neighbourhood of stressed [a], as in [ju ˈmaːvɐ] `I used to go', and for [ə] to occur in the environment of [e] or [ɛ], as in [ˈrwɛrəs] `Ruèras'. Because of this uncertainty, these two vowels will not be differentiated and both will be represented by <a>, respectively by /ɐ/.

In contrast, [e] and [ɛ] generally occur in stressed syllables in non-compound words, but in some loanwords they may occur in unstressed syllables, as is the case of the second [ɛ] in /\textit{ˈgɛnɛral}/ `general (adj.)'.

There are long and short vowels in Tuatschin, but minimal pairs contrasting long vowels to short vowels do not seem to exist. In unstressed syllables, only short vowels occur, but in stressed syllables, there are both short and long vowels. They will not be represented orthographically, but in the Tuatschin word list (chapter 9), all the lexical entries will be followed by a phonetic transcription indicating lengthening of the vowels.

Regarding [ʊ] and [u], \citet[130]{Liver2010} notes for Standard Sursilvan that [ʊ] mostly occurs in short syllables, whereas [u] mostly occurs in long syllables, with some exceptions. In Tuatschin, there is at least one minimal pair which opposes the two vowels in a short syllable: /ʥu/ `had' (participle of \textit{vaj} 'have') vs /ʥʊ/ `down'. The realisation of /ʊ/ varies between a nearly closed [u] and a very closed [o].

In my corpus, the close front rounded [y], represented by \textit{ü}, only occurs in recent loans from German or Swiss German. It is not included in \tabref{vow}. Examples are \textit{bürò} `office', \textit{mütòlògia} `mythology', and \textit{tüp} `person'.\footnote{The Sursilvan, Sutsilvan, and Surmiran varieties do not possess close-front and close-mid front rounded vowels except in recent German loans, in contrast to the Ladin varieties (Putér, Vallader, and Jauer) which possess /y/ and /ø/, as in \textit{tü} `you (\textsc{sg})' or \textit{magöl} `glass'. In Sursilvan /ø/ also exists in recent German loans: \textit{töf} `motorbike', which is pronounced \textit{téf} by old people.}

\tabref{vmp} presents some minimal pairs contrasting short vowels on the one hand, and contrasting long vowels on the other.

\begin{table}
	\caption{Vowel minimal pairs}
	\label{vmp}
	\begin{tabular}{llllllll}
	 \lsptoprule
		/i/&vs&/ɛ/& /fil/ &`thread' &vs& /fɛl/&`gall'\\
		/i/ & vs & /ɐ/ & \textit/ʃi/ & `so' & vs & /ʃɐ/ & `if'\\
		/e/&vs&/ɛ/& /leːʨ/ &`bed'&vs& /lɛːʨ/ &`marriage'\\
		&&&/me/ &`me' (dative) & vs & /mɛ/ & `me' (accusative) \\
		/ɛ/&vs&/ɔ/& /fɛl/ &`gall’ &vs & /fɔl/ & `bellows’\\
		&&& /sɛɲ/ &`sign’ & vs & /sɔɲ/ & `holy’\\
	
		/ʊ/ & vs & /ɔ/ & /kʊ/ & `how’ &vs& /kɔ/ &`here’\\
	&&&  /ˈrʊma/ & `Rome' &vs & /ˈrɔma/ & `branches'\\
	/ʊ/ & vs & /u/ &  /ʥʊ/ & `down' & vs & /ʥu/ & `had'\\
		/aː/ & vs & /ɛː/ &  /ʨaːr/ & `dear' & vs & /ʨɛːr/ & `expensive'\\
	/aː/ & vs & /ʊː/ & /kaːr/  & `bus' & vs & kʊːr & `heart'\\
		\lspbottomrule
	\end{tabular}
\end{table}


\section{Diphthongs}
In Tuatschin, diphthongs consist of the glides /j/ and /w/ as well as /i/ and /u/ in combination with /ə/. Tuatschin possesses five falling and ten rising diphthongs. \tabref{diph} shows the diphthongs with /j/ and /w/. Note that /ɔw/ is very rare and only occurs in the inverted forms of the first person singular present indicative forms of /saˈvaj/ `know' and /vaj/ `have': /sɔw/ `know I' and /vɔw/ `have I'. The following diphthongs do not occur: /ɔj/, /ʊj/, /uj/, /jʊ/, /ɛj/, /ɛw/, /ew/, /iw/, /ʊw/, /uw/, /wɔ/, /wʊ/, and /wu/.

The falling diphthong /uɐ/ is not very frequent. An example is  /ˈʃkuɐ/ `broom'.

Minimal pairs with diphthongs are rare in the corpus; there are only oppositions between /aj/ and /ej/, as in /najf/ `snow' vs /nejf/ `new', or /majl/ `apple' vs /mejl/ `honey'.

\begin{table}
\caption{Diphthongs}
	\label{diph}
	\begin{tabular}{llllll}
		\lsptoprule
		falling & & & rising & & \\
		\midrule
		/aj/ & /kwaj/ & `this' & /ja/ & /uˈjarɐ/ & `war'\\
		/ɛj/ & -- & & /jɛ/ & /ˈjɛdɐ/ & `time'\\
		/ej/ & /sejs/ & `their' & /je/ & /ˈjeli/ & `oil'\\
		/iɐ/ & /ˈʒbiɐr/ & `thug' & /jɐ/ & /ɐnˈʦjamɐn/ & `together'\\
		/ɔj/ & -- & & /jɔ/ & /kurˈjɔs/ & `strange'\\
		/uj/ & -- & & /ju/ & /ju/ & `I'\\
		/aw/ & /awn/ & `still' & /wa/ & /ˈawa/ & `water'\\
		/ɛw/ & -- & & /wɛ/ & /kwɛl/ & `this'\\
		/ew/ &  -- & & /we/ & /kwelm/ & `mountain'\\
		/iw/ & -- & & /wi/ & /kwɛlˈwizɐ/ & `in this way'\\
		/ɔw/ & /sɔw/ & `know I' \\
		\lspbottomrule
	\end{tabular}
\end{table}

The diphthong /aj/ is pronounced [aj] or [æj].

The difference between diphthongs and vowels in hiatus is not always straightforward. A spontaneous production of /piun/ `lard' with /i/ and /u/ in hiatus is found in (\ref{ex:lard}).

\ea\label{ex:lard}
\gll La mùm' ò méz ajn \textbf{piún}.\\
\textsc{def.f.sg} mother have.\textsc{prs.3sg} put.\textsc{ptcp.unm} into lard.\textsc{m.sg}\\
\glt `Mother added some lard.' (Sadrún, m5)
\z

But when asked whether \textit{piun} has one or two syllables, the consultant answered that it has only one syllable and pronounced it [pjun]. \citet[3f.]{Caduff1952} notes the same problem for the diphthong /iɐ/, which is sometimes pronounced [jɛ] or [je]. An example from the corpus is \textit{sté̱diamajn} `diligently', which is pronounced [ˌʃtediaˈmajn], but which also could also be pronounced [ˌʃtedjaˈmajn].

However, there are uncontroversial cases of hiatus, as e.g. /fuˈajna/, /uˈɔn/ `this year', or /uˈawt/ `forest', which are never pronounced /ˈfwajnɐ/, /wɔn/ or /wawt/, in contrast to \textit{uéstg} /wéʃʨ/ `bishop'.

Hiatus across word boundaries is usually avoided. There are two strategies. The first and generally used one is the elision of the last vowel of the first word if it is a weak vowel (\textit{ə} and \textit{ɐ}, both spelled <a>), as in \textit{Quaj \textbf{vèz' ò} uschéja} `This looks like that' instead of \textit{vèza ò}. Example (\ref{ex:sandhi1}) contains two examples of the weak vowel <a> which is elided (\textit{bigj' idéa} for \textit{bigja idéa} and \textit{stad' ajn} for \textit{stada ajn}), a well as one example of strong vowels that do not trigger elision (\textit{ju èra}).

\ea
\label{ex:sandhi1}
\gll Núa ṣè quaj hòtè̱l? \textbf{Bigj'} \textbf{idéa}, \textbf{ju} \textbf{èra} schòn ònṣ bigja \textbf{stad’} \textbf{ajn} quaj martgau.   \\
where \textsc{cop.prs.3sg} \textsc{dem.m.sg} hotel \textsc{neg} idea \textsc{1sg} be.\textsc{impf.1sg} already year.\textsc{m.pl} \textsc{neg} \textsc{cop.ptcp.f.sg} in \textsc{dem.m.sg} city\\
\glt `Where is this hotel? No idea, I hadn't been in that city for years.' (Ruèras, f7, l. 1665f.)
\z

The other, much less frequent strategy, is to insert an epenthetic \textit{n} between the two words as in (\ref{ex:euphonic1}), where \textit{n} is inserted between \textit{vasèva} and \textit{ins}.

\ea
\label{ex:euphonic1}
\gll    \textbf{Vasèva-n} \textbf{ins} ina signjura […] cun schuba cuérta, còtschna, […] lura spitgavan als purs ina grònda malaura […].\\
see.\textsc{impf.3sg-euph} \textsc{gnr} \textsc{indef.f.sg} woman {} with skirt.\textsc{f.sg} short red {} \textsc{corr} expect.\textsc{impf.3pl} \textsc{def.m.pl} farmer.\textsc{pl} \textsc{indef.f.sg} big storm\\
\glt `If one saw a woman with a short skirt, a red one, the farmers would expect a heavy storm.' (Sèlva, \citealt[34]{Büchli1966})
\z

This epenthetic \textit{n} tends to be used in written Standard Sursilvan; the more usual strategy would be elision like in \textit{vasèv' ins}.

 
\section{Consonants}
Tuatschin possesses 26 consonants which are presented in \tabref{cons}; consonant minimal pairs are shown in \tabref{cmp}.

\begin{table}
\caption{Consonants}
\label{cons}
 \begin{tabular}{llllllll}
  \lsptoprule
      &  & bilabial & labio-  & alveolar  &  palatal & palato- &velar\\
     &&& dental &&& alveolar \\
  \midrule
nasal    &    &  m   & &  n       &  	ɲ & & ŋ \\

stop &voiced   &  b  &   &  d     &  &  & ɡ\\
  & voiceless   &  p   &      & t  &  & & k\\
fricative  &  voiced  &      & v        & z & &	ʒ\\
  &  voiceless  &      &   f      & s & & ʃ & x, h\\
  affricate & voiced & & & & ʥ &ʣ& \\
  & voiceless &&& ʦ & ʨ & ʧ \\
trill  &    &      &         & r \\
lateral appr.  &    &      &         & l & ʎ \\
  \lspbottomrule
 \end{tabular}
\end{table}

The four consonants /h/, /x/, /ŋ/, and /ʣ/ have a restricted distribution and do not have a phonological status.

\begin{itemize}
	\item /h/ and /x/ only exist in Swiss German loans like  /halt/ `simply' or /rex/ `rich'
	
	\item /ŋ/ is an allophone of /n/ before /ɡ/ and /k/
	
	\item /ʣ/ is an allophone of /ʦ/ in word final position if it is followed by a vowel or a voiced consonant. In the corpus it occurs in a few words like /lɛʣ/ vs /lɛʦ/ `anaphoric demonstrative' and in /mjaʣˈde/ `noon (literally `half day')' vs /ˈmiaʦ/ `half \textsc{(m})'.
\end{itemize}

The other 21 consonants do have a phonological status as can be seen in \tabref{cmp}.

/l/ and /r/ have different realisations according to the speaker. Some speakers pronounce /l/ as [ł] although its distribution is not yet clear, and the uvular /ʁ/ is not uncommon among younger speakers.

Furthermore, /r/ and /l/ may have a syllabic realisation due to the dropping of [ɐ] between consonants as in [pr̩] vs [pɐr] `for' (§8.1, l. 47),  [ˌpr̩.mɐ.ˈvɛ.rɐ] vs [ˌpɐr.mɐ.ˈvɛ.rɐ] `Lent' §8.16, l.  1886), [pr̩.vɐ.zɛ.dɐrs] vs  pɐr̩.vɐ.zɛ.dɐrs] `herdsmen' (§8.6, l. 930). [tr̩] vs [tiɐr] `at, by' (§8.4, l. 747), [pr̩.ˈsu.lɐ] vs [pɐr.ˈsu.lɐ] (§8.4, l. 673) [ʃtr̩.mɐn.ˈtus] vs [ʃtɐr.mɐn.ˈtus] terrible (§8.2, l. 208) or [pl̩s] vs [pɐr ɐls] `for the' (§8.7, l. 1009).

The voiced stops /b/, /d/, and /ɡ/ are sometimes realised as voiceless consonants like in [ˈab̥ər] `but'; however, this is never the case with the voiced palatal fricatives /z/ and /ʒ/.

In rare cases, the voiceless stops show an aspirated realisation, as in [əmˈpʰaw] `a bit' (§8.4, l. 758) or [pʰas] `pass' (§8.7, l. 1026).

There are some cases of assimilation of consonants across word boundaries. In (\ref{ex:ass1}), \textit{détg di} `said to' is realised as [detːi], and in (\ref{ex:ass2}), \textit{détg: té} `said: you' is pronounced [detːe].


\ea\label{ex:ass1}
\gll [...] api vau \textbf{détg} \textbf{di} mùma [...].\\
{} and have.\textsc{prs.1sg.1sg} say.\textsc{ptcp.unm} \textsc{def.dat} mother\\
\glt `[...] I said to my mother [...].' (Sadrún, m4, l. 388)
\z

\ea\label{ex:ass2}
\gll Pi ò èla \textbf{détg}: «\textbf{Té} savèssaṣ í cul tat [...].\\
then have.\textsc{prs.3sg} \textsc{3sg} say.\textsc{ptcp.unm} \textsc{2sg} can.\textsc{cond.2sg} go.\textsc{inf}  with.\textsc{def.m.sg} grandfather\\
\glt `Then she said: “You could go up with your grandfather [...].' (m4, l. 392f.)
\z

There are also some cases where complex consonant clusters are avoided, like for instance [ʨs], which is sometimes pronounced [ʦ] like in [maʦ] instead of [maʨs] `bunches' (§8.11, l. 1438f.).

In rapid speech /ʨ/ may be realised as an unreleased consonant, yielding a sound that is close to the glide /j/. Examples are \textit{létg/léj} `bed'  §8, l. 560), \textit{hanlétg/hanléj} `business' (§8.9, l. 1216), or \textit{atgnamajn/ajnamajn} `actually'(§8.16, l. 2013).

A major problem in analysing consonants is the question whether the voiceless word final consonants should be considered as such or as underlyingly voiced. For instance, in Standard Sursilvan the 1st person singular conditional is written \textit{cantass} `I would sing', but in Tuatschin when followed by a vowel or a voiced consonant, it is pronounced /z/, as in \textit{stèṣ} /ʃtɛz/ (\ref{ex:stes}).

\ea\label{ex:stes}
\gll [...] álṣò sch' ju \textit{\textbf{stèṣ}} aun fá in' jèda quaj, \textit{\textbf{figès}} ju bétga.\\
{} well if \textsc{1sg} must.\textsc{cond.1sg} still do.\textsc{inf} one.\textsc{f.sg} time \textsc{dem.unm} do.\textsc{cond.1sg} \textsc{1sg} \textsc{neg}\\
\glt `[...] well, if I had to do it once more, I wouldn't do it.' (Sadrún, m10, l. 1060f.)
\z

Other examples are `not even' and `eight'. `Not even' is spelled \textit{gnanc} in Standard Sursilvan, but in Tuatschin it is pronounced with [ɡ] instead of [k] if it is followed by a vowel or a voiced consonant, like in /ɲaŋg in/ `not even one'. `Eight' is a similar case since it is spelled \textit{otg} in Standard Sursilvan, but if it is followed by a vowel or a voiced consonant it is pronounced [ɔʥ] like in [ɔʥ ɔns] `eight years' (§8.15, l. 1819f.). However, in \textit{òtgònta} `eighty' \textit{òtg} is pronounced [ɔʨ] and not [ɔʥ]. 

If all final voiceless consonants were pronounced voiced when followed by a voiced element, it would be very easy to establish the rule that every voiceless consonant in word final position is pronounced voiced if followed by a vowel or a voiced consonant. However, this is not the case. In /in brˈiɐk əd in kʊp/ `a wooden bucket and a bowl' /k/ is never pronounced [ɡ]: /in *brˈiɐɡ əd in kʊp/.  The same holds for \textit{ljuc} `place': /in ʎuk ˌɐmpɐrˈnajval/ `a cosy place' vs /in *ʎuɡ ˌɐmpɐrˈnajval/. A further example is /fjuk/ `fire' whose diminutive is /fjukɛt/ and not /*fjuɡɛt/.

Therefore I will deviate from the Standard Sursilvan spelling and write voiced consonants in case they are pronounced voiced when followed by a voiced element, and the rule will be formulated as follows:

\begin{quote}
Every word final voiced consonant is pronounced as voiceless in isolation or preceding a word starting with a voiceless consonant, whereby \textit{j}, \textit{m}, and \textit{n} behave like voiceless consonants.
\end{quote}

That the nasals behave like voiceless consonants is demonstrated by the opposition between \textit{gjuvan} /ˈʥuvɐn/ `young (\textsc{m.sg}) v. \textit{gjufna} /ˈʥufnɐ/ `young (\textsc{f.sg})'.

The consequences for the spelling system used in this book will be discussed in §2.5.

\begin{table}
\caption{Consonant minimal pairs}
\label{cmp}
 \begin{tabular}{llllllll}
 \lsptoprule
p&vs&t& /pawn/&`bread'&vs& /tawn/ &`so much'\\
p&vs&n& /pawk/&`little (quant.)'&vs& /pawn/ &`bread' \\
p&vs&ɳ& /kʊp/&`bowl'&vs&\/kʊɳ/ & `wedge' \\
p&vs&s& /pawn/&`bread'&vs& /sawn/ & `blood' \\
p&vs&ʨ&\ /pawn/&`bread'&vs& /ʨawn/ & `dog' \\
t&vs&k&  /bʊt/&`barrel'&vs& /bʊk/ & `billy goat' \\
t&vs&n& /sɐˈlit/&`greeting'&vs& /sɐˈlin/ & `wheat' \\
k&vs&f& /ʥuk/&`play'&vs&\textit{ʥuf}&`yoke'\\
b&vs&n& /ˈrawbɐ/&`merchandise'&vs& /ˈrawnɐ/ & `frog'\\
d&vs&n&\ /ˈfrːidɐ/&`wound'&vs&/ˈfriːnɐ/ & `flour'\\
d&vs&ʦ& /ˈsɛndɐ/&`path'&vs& /ˈsɛnʦɐ/ & `without'\\
ɡ & vs & l & /nɐˈɡaː/ & `drown' & vs & /lɐˈvaː/ & `wash'\\
m&vs&l& /fɔm/ & `hunger' & vs & /fɔl/ & `bellows'\\
&&& fiˈ/maː/& `smoke' & vs& /fiˈlaː/& `spin'\\
n&vs&ɲ& /ɔn/&`year'&vs&\/ɔɲ/ &`alder'\\
n&vs&ʦ& /pʊn/&`bridge'&vs&\/pʊʦ/&`pond'\\
n&vs&ʨ& /lɛn/&`firewood'&vs&/lɛʨ/&`marriage'\\
ɲ&vs&ʨ&/peɲ/&`fir tree'&vs&/peʨ/ & `pick'\\
ɲ&vs&ʦ& pɛɲ/&`pledge'&vs&/pɛʦ/&`chest'\\
f&vs&r&  /najf/ &`snow'&vs&\/najr/ &`black'\\
s&vs&l&  /pas/&`step'&vs&\/pal/&`post'\\
ʃ&vs&ʨ& /eʃ/&`door'&vs &\/eʨ/&`ointment'\\
ʨ&vs&ʦ& /ɔʨ/&`eight'&vs&/ɔʦ/&`today'\\
ʨ&vs&ʧ& /deʨ/&`said'&vs&/deʧ/&`(I) say'\\
ʥ&vs&l& /ʥuf/&`yoke'&vs&/luf/ & `wolf'\\
  \lspbottomrule
 \end{tabular}
\end{table}

\section{Syllable structure}
Tuatschin possesses open and closed syllables, short and long. Long syllables (represented by VV in the table below) are realised by long vowels as well as by diphthongs. Consonant clusters occur in onset and coda position in a restricted way. The combination of two consonants in onset position is realised by a voiced or voiceless stop and /l/ or /r/; three consonants in onset position correspond to the same combinations but preceded by /ʃ/. In coda position, only two consonants occur (/rɛʃt/ `leftover'), but if the plural suffix \mbox{\textit{-s}} is added, three syllables also occur (/rɛʃts/ `leftovers'). Some syllable types are presented in \tabref{syllt}.

\begin{table}
	\caption{Syllable types}
	\label{syllt}
	\begin{tabular}{lll}
		\lsptoprule
V &	/ɐ/ &`and'\\
VV & /iː/ & `go'  \\
& /aj/ &`expletive pronoun'\\
VVC & /eːr/ & `yesterday\\
& /ejf/ & `egg'\\
VVCC & /welp/ & `fox'\\
CV & /ʥʊ/ & `down'\\
CVC & /tup/ & `stupid'\\
CVCC & //ʨern/ & `horn'\\
CVCCC & /tɛkst/ & `text'\\
CVV & /diː/ & `say'\\
& /baw/ & `beetle'\\
CVVC& /majns/ & `month'\\
CVVCC & /nejfs/ & `nephew'\\
CCVC & /tras/ & `through'\\
CCVCC & /krɛʃt/ & `hill'\\
CCVV & /kraj/ & `believe'\\
& /praw/ & `meadow'\\
CCVVC & /trʊːʨ/ & `narrow path'\\
& /plajd/ & `word'\\
CCVVCC & /trwiʎs/ & `narrow paths'\\
CCCVCVV & /ʃpriˈʦaː/ & `squirt'\\
CCCVC & /ʃtrɔm/ & `straw'\\
CCCVVC & /ʃtruːʃ/ & `almost'\\
CCCVCC & /ʃtrɛnʨ/ & `strict'\\
V.CCV & /ˈɔ.vrɐ/ & work'\\
VC.CV & /ˈɔl.mɐ/ & `soul'\\
V.CVC & /i.ˈral/ & `threshing floor'\\
CV.CCVC & /sa.ˈblun/ & `sand'\\
CV.CVVC & /fɐˈliɐn/ & `spider'\\
CV.CV.CCV.CV & /fi.li.ˈʃtʊ.kɐ/\\
CVC.CV.CV.CV.CVVC & /kɔn.fɐ.dɐ.ra.ˈʦjun/ & `confederation\\
CVC.CV.CV.CVV.CVC & /ma.lsɐ.ʥ.iˈdaj.vɐl/ & `ungainly'\\
 \lspbottomrule
\end{tabular}
\end{table}

 Tuatschin possesses words of one, two, and three syllables. Words with four or five syllables are rare and are usually compound words (/{ʥu.vɐn.ˈteʨ.nɐ/ `youth' or /ˌɔː.rɐ.ʥʊ.ˈzʊt/ `underneath in direction down the valley'). More examples of polysyllabic words are to be found in \tabref{syllt}.
 	
 	Stress is fixed and can be placed on the ultimate (/ɐn.ˈtʊrn/ `around' or /ɐn.ʦɐr.ˈdaː/ `aerate the soil'), the penultimate (/mu.ˈvɐl/ `cattle'), or the ante-penultimate syllable, which is rare (/ʦɐ.ˈɡrin.dɐ.rɐ/ `Yenish woman').

An in-depth analysis of stress, especially of the distribution of secondary stress, must be left to further studies.

\entry{rèst}{rɛʃt}{m.n}{rest. \textbf{\textit{rèsts}} leftovers}

\section{Spelling system}
The spelling system used in this grammar is a compromise between Standard Sursilvan spelling and the aim of making pronunciation and word stress transparent to the reader, which means that one grapheme has to correspond to one sound (or phoneme in most cases) (see \tabref{graphIpaI} and \tabref{graphIpaII}). 

The problems that Standard Sursilvan spelling does not solve are

\begin{itemize}
\item whether <e> and <o> are close-mid or mid, 
\item whether two adjacent vowels form a rising or falling diphthong or whether they represent two vowels in hiatus,
\item whether <s> and <sch> are voiced or not, 
\item and, in some cases, on which syllable of a given word stress falls.
\end{itemize}
 
To disambiguate these problems, I will indicate with an acute accent <é, ó> that the vowel is close-mid, and with a grave accent that the vowel is mid (<è, ò>) or near close (<ù>). Voiced palatal fricatives get a dot under the \textit{s} (<ṣ> for /z/ and <ṣch> for /ʒ/), as is the usage in Romansh bilingual dictionaries. The other cases will be explained below.

Stress rules are as follows:

\begin{itemize}
	\item Diphthongs are always stressed (\textit{autar} /ˈawtɐr/ `other').
	\item Words without a diphthong which end in a vowel or <-n> or <-s> are stressed on the penultimate syllable (\textit{ˈtata} /ˈtatɐ/ `grandmother', \textit{anzjaman} /ɐnˈʦjamɐn/ `together', \textit{casas} /ˈkazɐs/ `houses').
	\item Words without a diphthong ending in a consonant, except for words ending in <-n> or <-s>, are stressed on the last syllable (\textit{racrut} /rakˈrut/ `recruit').
	\item Mid and close-mid vowels (<é, è, ó, ò>) are stressed except if they occur in a word containing a diphthong.
	\item Words ending in a vowel or <n> or <s> which are stressed on a syllable other than the penultimate get an acute accent. This concerns the vowels \textit{a}, \textit{i}, and \textit{u}. In other words, \textit{á}, \textit{í}, and \textit{ú} are always stressed (\textit{cantá} /kɐnˈtaː/ `sing', \textit{barcún} /bɐrˈkun/ `shutter', \textit{barbís} /bɐrˈbiːs/ `moustache', \textit{fugí} /fuˈʥiː/ `flee').
	\item Words with two ditphthongs, with two vowels with diacritics, and words with a diphthong and a vowel with a diacritic get an underscore under the stressed vowel or diphthong (\textit{èxtrè̱m} /ɛksˈtrɛːm/ `extreme', \textit{gròndè̱zja} /grɔnˈdɛzjɐ/ `size').
	\item Other cases of stressed vowels which are not covered by the preceding rules also get an underscore, as for example \textit{mè̱ndar} /ˈmɛndɐr/ `worse', \textit{anavù̱s} /ɐnɐˈvʊːs/ `back, backward', \textit{antù̱rn} /ɐnˈtʊrn/ `around'.
\end{itemize}


The reason for giving <-n> and <s> special treatment is the fact that <-n> is used for verbal plural and <-s> for nominal and verbal plural. If <-n> and <-s> were treated like the other final consonants, many more diacritics would be used.

A further problem is the treatment of <s> followed by a consonant. Here, I follow the Standard Sursilvan spelling:

\begin{itemize}
	
	\item <s> followed by <c\footnote{Before consonants as well as before vowels other than \textit{e} and \textit{i}.}, f, m, n, p, qu, r, t> is pronounced [ʃ].
	\item <s> followed by <b, d, g, v> is written <ṣ> and pronounced [ʒ].
	\item <s> followed by <c\footnote{Before \textit{e} or \textit{i}.}, l, n, z is pronounced [s].
	\item If <s> should not be pronounced [ʃ] or [ʒ], a hyphen separates the two consonants as in <cuns-cianza> (/kunsˈʦiɐnʦɐ/) `conscience', <ris-plí> (/risˈpli/) `pencil', or <mjɐdṣ-dé/mjɐṣ-dé> (/mjɐdz-de, mjɐz-de) `noon'. A hyphen is also used to separate <g> from <j> in order to prevent this combination from being pronounced /ʥ/, as e.g. in \textit{nag-janta} `(s)he drowns'.
\end{itemize}

In the texts (chapter 8) and hence also in the examples taken from these texts, the final consonants are transcribed as they are pronounced. An example is the word for `ten', which can be transcribed [déjʃ] or [déjʒ] according to the context in which it occurs.

\tabref{graphIpaI} and \tabref{graphIpaII} present the correspondences between the spelling used in this grammar and the IPA phonetic alphabet.


\begin{table}
\caption{Correspondences between spelling and IPA I}
\label{graphIpaI} 
\begin{tabular}{lll}
    \lsptoprule
        grapheme      & IPA\\
    \midrule  
  a & ə, ɐ, a\\
  á & a\\
  b & b\\
  c & k before a, ó, ò, ù, u\\
  & ʦ before é, è, i\\
  ch & k before é, è, i\\
  d & d\\
  dṣ & ʣ\\
  é & e\\
  è & ɛ\\
  f & f\\
  g & ɡ before a, ó, ò, ù, u\\
  & ʥ before é, è, i\\
  gh & ɡ before é, è, i\\
  gj & ʥ before a, ó, ò, ù, u\\
  gl & ʎ before i and word finally\\
  & ɡl before a, é, è, ó, ò, ù, u\\
  glj & ʎ before a, è, é, ò, ó, ù, u\\
  gn& ɲ\\
  h & x, h\\
  i & i\\
  j & j\\
  l & l\\
  l̩ & l̩\\
  m & m\\
  n & n\\
  ó & o\\
  ò & ɔ\\
  p & p\\
  qu & kw\\
  r & r, ʁ\\
  r̩ & r̩\\
  s & s word initially, word finally, and preceding <l, n, z>\\
  & z between two vowels\\
  & ʃ before <c, f, m, n, p, qu, r, t>\\
  ss & s between two vowels\\
  
  \lspbottomrule
\end{tabular} 
\end{table}


\begin{table}
\caption{Correspondences between spelling and IPA II}
\label{graphIpaII} 
\begin{tabular}{lll}
    \lsptoprule
        grapheme      & IPA\\
    \midrule  
 ṣ & z\\
 & ʒ preceding <b, d, g, v>\\
  sch & ʃ\\
  ṣch & ʒ\\
  t & t\\
  tg& ʨ\\
  tsch & ʧ\\
  u & u\\
  ù & ʊ\\
  v & v\\
  x & ks\\
  z & ʦ before a, ó, ò, ù, u\\
  \lspbottomrule
\end{tabular} 
\end{table}

Furthermore, the following orthographic signs are used to indicate elided vowels: ' for vowels elided at the end of a word, as in \textit{l'agid} `the help' (instead of \textit{l\textbf{a} agid}), and ` for vowels elided at the beginning of a word, as in \textit{ò `l détg} `has he said' (instead of \textit{ò \textbf{è}l détg}).

