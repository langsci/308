\chapter{Simple sentences}

\section{Declarative sentences}


\subsection{Order of the arguments}
Tuatschin has verb-second syntax, which means that if no element of the sentence occurs before the subject, the subject precedes the verb as in (\ref{ex:sov1}).

\ea
\label{ex:sov1}
	\gll    [...] álṣò, {\ob}nuṣ{\cb} {\ob}vajn gju{\cb} {\ob}ina pintga piaglja{\cb} [...].\\
 {} well \textsc{1pl} have.\textsc{prs.1pl}  have.\textsc{ptcp.unm} \textsc{indef.f.sg} small salary \\
\glt `[...] well, we got a small salary [...].' (Sadrún, m6, 1. 1301f.)
\z

As shown in §4.2.2 above, the indirect object usually precedes the direct object (\ref{ex:indir1}).

\ea
\label{ex:indir1}
\gll    Quèl vès lu aun da pijè {\ob}da té{\cb} {\ob}al pustrètsch dal piartg tga té vèvas partgirau{\cb}. \\
\textsc{dem.m.sg} have.\textsc{cond.3sg} then still \textsc{comp} pay.\textsc{inf} \textsc{dat} \textsc{2sg} \textsc{def.m.sg} money of.\textsc{def.m.sg} pig \textsc{rel} \textsc{2sg} have.\textsc{impf.2sg} look\_after.\textsc{ptcp.unm}\\
\glt `This one should still pay you the money for the pig you had looked after.' (Sadrún, m6, 1. 1414ff.)
\z

If there is an element that precedes the subject, the subject is moved after the finite verb as in (\ref{ex:sov2}) and (\ref{ex:sov4}).

\ea
\label{ex:sov2}
\gll {\ob}\textbf{Avaun} \textbf{in} \textbf{pèr} \textbf{jamnas}{\cb} … èr’ {\ob}\textbf{ju}{\cb} gjù Locarno [...].\\
before \textsc{indef.m.sg} couple week.\textsc{f.pl} {} \textsc{cop.impf.1sg} \textsc{1sg} down \textsc{pn}\\
\glt `A couple of weeks ago … I was in Locarno [...].' (Sadrún, m8, 1. 1431)
\z

\ea
\label{ex:sov4}
\gll  [...] a {\ob}\textbf{sjantar}{\cb} va {\ob}{\textbf{ju}}{\cb} … cumprau in zacù̱n.\\
{} and after  have.\textsc{prs.1sg} \textsc{1sg} {} buy.\textsc{ptcp.unm}  \textsc{indef.m.sg} second\\
\glt `[...] but afterwards I … bought a second [one].' (Ruèras, m10, 1. 992f.)
\z


The coordinating conjunctions \textit{ábar} `but' (\ref{ex:sov3}) and \textit{dantaun} (\ref{ex:sov5}) `however' do not trigger subject inversion. In contrast, \textit{api} and its shorter form \textit{pi} `and, and then' do trigger it (\ref{ex:sov3}, \ref{ex:aapi} and \ref{ex:aapi1}).

\ea
\label{ex:sov3}
\gll \textbf{Api} \textbf{vau} détg èls vágian fatg ina tura tschèl’ jamna, \textbf{ábar} \textbf{èl} vaj quitau tga quaj séj … .\\  
and\_then have.\textsc{prs.1sg} say.\textsc{ptcp.unm} \textsc{3pl.m} have.\textsc{prs.sbjv.3pl} make.\textsc{ptcp.unm} \textsc{indef.f.sg} tour  \textsc{dem.f.sg} week but \textsc{3sg.m} have.\textsc{prs.sbjv.3sg} worry.\textsc{m.sg} \textsc{comp} \textsc{dem.unm} \textsc{cop.prs.sbjv.3sg} \\
\glt `And then I said that they had done a tour that week, but that he sees to it that this be … .' (Sadrún, f3, 1. 114ff.)
\z

\ea\label{ex:sov5}
\gll    Ju vagnès schòn, \textbf{dantaun} \textbf{vuṣ} \textbf{vajs} taun tschùf ajn tgèsa.\\
\textsc{1sg} come.\textsc{cond.1sg} really however \textsc{2pl} have.\textsc{prs.2pl} so\_much dirt in house\\
\glt `I would come, of course, but you have so much dirt in your house.' (\citealt[18]{Berther1998})
\z

\ea
\label{ex:aapi}
\gll    Als gjanitu̱rs fagèvan al pur, \textbf{ad} \textbf{ju} \textbf{èra} tschavrèr’ in tjams, ajn l’ antschata cun mju frá, in òn parsula, \textbf{api} \textbf{stavèv’} \textbf{ins} í culas tgauras tòca sé Nalps [...]. \\
\textsc{def.m.pl} parent.\textsc{pl} do.\textsc{impf.3pl} \textsc{def.m.sg} farmer and \textsc{1sg} \textsc{cop.impf.1sg} goatherd.\textsc{f.sg} \textsc{indef.m.sg} time in \textsc{def.f.sg} beginning with \textsc{poss.1sg.m.sg} brother one.\textsc{m.sg} year alone.\textsc{f.sg}  and  must.\textsc{impf} \textsc{gnr} go.\textsc{inf} with.\textsc{def.f.pl} goat.\textsc{pl} until up  \textsc{pn} \\
\glt `My parents were farmers, and I was a goatherd for a certain time, at the beginning with my brother, one year alone, and one had to go with the goats as far as Nalps [...].' (Sadrún, f5, 1. 1253ff.)
\z

\ea
\label{ex:aapi1}
	\gll   \textbf{Pi} ṣè `\textbf{l} vajramájn staus lò. \\
	then be.\textsc{prs.3sg} \textsc{3sg.m} true.\textsc{f.sg.adv} remain.\textsc{ptcp.m.sg} there\\
\glt `And it [= the swan] really stayed there.' (Sadrún, f8, l. 1457)
\z

After subordinating conjunctions there is no subject inversion (\ref{ex:subconj1}--\ref{ex:subconj3}).

\ea
\label{ex:subconj1}
\gll    La buéba ò détg \textbf{tg'} \textbf{èla} \textbf{ségi} bétga \textbf{ida} vòluntárjamajn, èla stuévi ira.\\
\textsc{def.f.sg} girl have.\textsc{prs.3sg} say.\textsc{ptcp.unm} \textsc{comp} \textsc{3sg} be.\textsc{prs.sbjv.3sg} \textsc{neg} go.\textsc{ptcp.f.sg} voluntary.\textsc{adv} \textsc{3sg.f} must.\textsc{impf.sbjv.3sg} go.\textsc{inf}\\
\glt `The girl said that she didn't go voluntarily, [but that] she was obliged to go.' (Bugnaj, \citealt[132]{Büchli1966})
\z

\ea
\label{ex:subconj2}
\gll  [...] api lura va ju in’ jèda talafònau dad èl, \textbf{pr̩quaj} \textbf{tg'} \textbf{èl} \textbf{vèva} \textbf{tr̩mèz} in’ anunzja da mòrt [...]\\
{} and then have.\textsc{1sg}  \textsc{1sg} one.\textsc{f.sg} time call.\textsc{ptcp.unm} \textsc{dat} \textsc{3sg.m} because \textsc{comp} \textsc{3sg.m} have.\textsc{impf.3sg} send.\textsc{ptcp.unm} \textsc{indef.f.sg} announcement of death.\textsc{f.sg}\\ 
\glt `[...] and then I phoned him once, because I should send a death notice [...].' (Sadrún, f3, 1. 18ff.)
\z

\ea
\label{ex:subconj3}
\gll  [...] \textbf{avaun} \textbf{c’} \textbf{ju} \textbf{sùn} \textbf{staus} tial tat savèvu da quaj nuét [...].\\
{} before \textsc{subord} \textsc{1sg} be.\textsc{prs.1sg} \textsc{cop.ptcp.m.sg} at.\textsc{def.m.sg} grandfather know.\textsc{impf.1sg.1sg} of \textsc{dem.unm} nothing\\
\glt `[...] before I stayed with my grandfather I didn’t know anything about that [...].' (Sadrún, m4, 1. 336f.)
\z

If a subject which would normally be inverted is focalised, it may occur preverbally, as \textit{quèls} `these' in (\ref{ex:subconj4}).

\ea
\label{ex:subconj4}
\gll   A lur scha `l vèṣ ussa pagljau par èxè̱mpal in grép tga vès pudju bétar èl, scha \textbf{quèls} fùssan grat schulaj gjù ajl’ aua, ajl lac.\\
and then if \textsc{3sg.m} have.\textsc{cond.3sg} now hit.\textsc{ptcp.unm} for example.\textsc{m.sg} \textsc{indef.m.sg} rock \textsc{rel} have.\textsc{cond.3sg} can.\textsc{ptcp.unm} throw.\textsc{inf} \textsc{3sg.m} \textsc{corr} \textsc{dem.m.pl} be.\textsc{cond.3pl} immediately fall\_rapidly.\textsc{ptcp.m.pl} down into.\textsc{def.f.sg} water into.\textsc{def.m.sg} lake\\
\glt `And then if it [the load] hit a rock which could have thrown it down, these [= the mules] would have immediately fallen down rapidly into the water, into the lake.' (Ruèras, m10, 1. 1066ff.)
\z

The inverted subject is not immediately adjacent to the verb, and in (\ref{ex:sagir1}) \textit{sagir} `sure' is located immediately after the verb.

\ea
\label{ex:sagir1}
	\gll [...] avaun nus \textbf{èra} \textbf{sagir} \textbf{al} \textbf{tgavrè} èra schòn jus culas tgauras, lèz mava lu èra.\\
{} before \textsc{1pl} be.\textsc{impf.3sg} sure \textsc{def.m.sg} goatherd also already go.\textsc{ptcp.m.sg} with.\textsc{def.f.pl} goat.\textsc{pl} \textsc{dem.m.sg} go.\textsc{impf.3sg} then also\\
\glt `[...] before us the goatherd had certainly already gone with the goats, he also used to go.' (Sadrún, m6, 1. 1317f.)
\z

However, some consultants reject the position of \textit{sagir} in (\ref{ex:sagir1}). For them, \textit{sagir} should be located between \textit{èra} `also' and \textit{schòn} `already'.

The subject pronoun is usually obligatory in Tuatschin, but the omission of the subject may occur in subordinated and coordinated clauses if it can be recovered from the context (\ref{ex:nosubj1} and \ref{ex:nosubj2}).

\ea
\label{ex:nosubj1}
\gll    Avaun ina fjasta mavan \textbf{aj} … tialas … gjufnas … par nègla, partgé \longrule{} matévan sé sé la capjala … ina nègla.\\
before \textsc{indef.f.sg} celebration go.\textsc{impf.3pl} \textsc{3pl} {} to.\textsc{def.f.pl} {}  young\_woman.\textsc{pl} {} for carnation.\textsc{f.pl} because \textsc{sbj}  put.\textsc{impf.3pl} up up  \textsc{def.f.sg} hat {} \textsc{indef.f.sg} carnation \\
\glt `Before a celebration they would go … to the … girls for carnations, because they would put … a carnation on their hat.' (Zarcúns, m2, 1. 1567ff.)
\z

\ea
\label{ex:nosubj2}
 \gll    Al sulèt intarassánt è l’ ampréma sacùnda classa \textbf{nùca} \textbf{tga} \longrule {} fòn la midada tial sursilván […].\\
 \textsc{def.m.sg} only interesting \textsc{cop.prs.3sg} \textsc{def.f.sg} first second form where \textsc{rel} \textsc{sbj} make.\textsc{prs.3pl} \textsc{def.f.sg} change towards.\textsc{def.m.sg} Sursilvan\\
 \glt `The only interesting thing is the first [and] second form where they [the pupils] start switching towards Sursilvan [...].' (Sadrún, m5)
 \z
 
 
 \ea
 \
 \gll Lò bagagjávani gjù catschina a barṣchavan \textbf{grad} èla.   \\
 there build.\textsc{impf.3pl.3pl} down limestone.\textsc{f.sg} and  burn.\textsc{impf.3pl} immediately \textsc{3sg.f}\\
\glt `There they would mine limestone and burn it immediately.' (Sadrún, m4, l. 420f.)
\z
 
 

\section{Interrogative sentences}
Polar questions are characterised by a rising intonation and subject inversion (\ref{ex:interr1}).

\ea
\label{ex:interr1}
\gll Gè, sùnd ju ajn tju taritòri, distùrb’ ju té?   \\
yes \textsc{cop.prs.1sg} \textsc{1sg} in \textsc{poss.2sg.m.sg} territory disturb.\textsc{prs.1sg} \textsc{1sg} \textsc{2sg} \\
\glt `Yes, am I in your territory, do I disturb you?' (Sadrún, m8, l. 1447)
\z

Content questions require the presence of an interrogative word and, like polar questions, exhibit subject inversion. Interrogative pronouns are \textit{cu/cura} `when' (\ref{ex:int4}), \textit{cù} `how' (\ref{ex:int8}), \textit{dacù} `why' (\ref{ex:int10}), \textit{danùndar} `where from' (\ref{ex:int11}), \textit{núa} `where' (\ref{ex:int3}), \textit{partgé(j)} `why' (\ref{ex:int5}), \textit{tgé(j)} `what' (\ref{ex:int2}), \textit{tgé(j)nín/tgé(j)nina} `which one'(\ref{ex:int7}), \textit{tgi} `who' (\ref{ex:int1}). The interrogative determiner is \textit{tgé(j)} `which, what'(\ref{ex:int6}).

\ea
\label{ex:int4}
\gll \textbf{Cu} ṣè la vagnida?\\
when be.\textsc{prs.3sg} \textsc{3sg.f} come.\textsc{ptcp.f.sg}\\
\glt `When did she come?' (Sadrún, m5)
\z

\ea
\label{ex:int8}
\gll \textbf{Cù} détsch ju? Gjù Sardégna?\\
how say.\textsc{prs.1sg} \textsc{1sg} down \textsc{pn}\\
\glt `How do I say? Down Sardinia?'(Ruèras, f7, l. 1685)
\z

\ea
\label{ex:int10}
\gll \textbf{Dacù} as té fatg quaj?\\
how have.\textsc{prs.2sg} \textsc{2sg} do.\textsc{ptcp.unm} \textsc{dem.unm}\\
\glt `Why did you do this?'(own knowledge)
\z

\ea
\label{ex:int11}
\gll \textbf{Danùndar} èn èls vagní?\\
where\_from be.\textsc{prs.3pl} \textsc{3pl} come.\textsc{ptcp.m.pl}\\
\glt `Where did they come from?'(own knowledge)
\z

\ea
\label{ex:int3}
\gll \textbf{Núa} ajs staus?\\
where be.\textsc{prs.2sg} \textsc{cop.ptcp.m.sg}\\
\glt `Where have you been?' (Sadrún, m5)
\z

\ea
\label{ex:int5}
\gll \textbf{Pr̩tgéj} fas quaj?\\
why do.\textsc{prs.2sg} \textsc{dem.unm}\\
\glt `Why do you do this?' (Sadrún, m5)
\z

\ea
\label{ex:int2}
\gll \textbf{Tgéj} lajn fá avaun c' í a raschlá?\\
what want.\textsc{prs.1pl} do.\textsc{inf} before \textsc{rel} go.\textsc{inf} \textsc{comp} rake.\textsc{inf}\\
\glt `What shall we do before going to rake?' (Cavòrgja, \citealt[121]{Büchli1966})
\z

\ea
\label{ex:int6}
\gll \textbf{Tgéj} \textbf{tgamiṣcha} as cumprau?\\
which shirt.\textsc{f.sg} have.\textsc{prs.2sg} buy.\textsc{ptcp.unm}\\
\glt `Which shirt did you buy?' (Sadrún, m5)
\z

\ea
\label{ex:int7}
\gll \textbf{Tgéjnina} as cumprau?\\
which\_one have.\textsc{prs.2sg} buy.\textsc{ptcp.unm}\\
\glt `Which one did you buy?' (Sadrún, m5)
\z

\ea
\label{ex:int1}
\gll \textbf{Tgi} ò malagjau quaj malètg?\\
who have.\textsc{prs.3sg} paint.\textsc{ptcp.unm} \textsc{dem.m.sg} picture\\
\glt `Who painted this picture?' (Sadrún, m5)
\z

Indirect interrogative clauses will be treated in §6.2.10 below.


\section{Imperative sentences}
Imperative sentences may or may not lack a subject. If they have one, it may precede (\ref{ex:imp.precede}) or follow the verb (\ref{ex:imp.follow}).

\ea
\label{ex:imp.precede}
\gll «\textbf{Té} \textbf{nò} lu vidòr ússa. Lò, quèsta sèra \textbf{dòrma} lu \textbf{bigja} ajn lò.»\\
\textsc{2sg} come.\textsc{imp.2sg} then down now there  \textsc{dem.f.sg} evening sleep.\textsc{imp.2sg} then \textsc{neg} in there\\
\glt `Come down here now. There, don’t sleep up there this evening.» (Sadrún, m4, l. 616f.)'
\z

\ea
\label{ex:imp.follow}
	\gll  Ah uòn \textbf{nò} \textbf{té} ajnta Pardatsch.\\
eh this\_year \textsc{go.imp.2sg} \textsc{2sg} into \textsc{pn}\\
\glt `Ah, this year go to Pardatsch.' (Cavòrgja, m7, l. 2190f.)
\z

There are some more elements that may precede an imperative: an indirect object (\ref{ex:impdat}) and the adverb \textit{mù} `only, just' (\ref{ex:impmu}).

\ea
\label{ex:impdat}
\gll Gè, \textbf{da} \textbf{mé} daj è ina … . \\
yes \textsc{dat} \textsc{1sg} give.\textsc{imp.2sg} also \textsc{indef.f.sg} \\ 
\glt `Yes, give me also a ... .' (Ruèras, m3, l. 2020)
\z

\ea
\label{ex:impmu}
\gll [...] \textbf{mù} spétga … .\\
{} only wait.\textsc{imp.2sg}\\
\glt `[...] just wait.' (Ruèras, m3, l. 2090)
\z


In prohibitive sentences, the negator \textit{bétga} and its allomorphs precede the imperative (\ref{ex:betga1}). But if some elements precede the imperative as in (\ref{ex:imp.precede}), the negator follows it.

\ea
\label{ex:betga1}
\gll  \textbf{Bétg} tumaj!\\
\textsc{neg} be\_afraid.\textsc{imp.2pl}\\
\glt `Don’t be afraid!' (\DRGoK{2}{503})
\z

Reflexive verbs use the prefix \textit{sa-} in the imperative (\ref{ex:imprefl}) as in all other tenses, moods, and non-finite categories.

\ea
\label{ex:imprefl}
\gll  \textbf{Sapartgiraj} dals bètlars cu tgi vòn a tgavaj.  \\
\textsc{refl}.beware.\textsc{imp.2pl} of.\textsc{def.m.pl} beggar.\textsc{pl} when \textsc{rel.3pl} go.\textsc{prs.3sg} on horse.\textsc{m.sg}\\
\glt `Beware of the beggars when they ride.' (\DRGoK{2}{327})
\z

If a stative verb forms an imperative, subjunctive mood is used, like \textit{vajas} `have subjunctive \textsc{2pl}' in (\ref{ex:impsubj}).

\ea
\label{ex:impsubj}
\gll    \textbf{Vajas} quitaus cu vus majṣ ṣur la lingja via … dal dṣùc, dal zùc.»\\
have.\textsc{imp.2pl} worry.\textsc{m.pl} when \textsc{2pl} go.\textsc{prs.2pl} over \textsc{def.f.sg} line over of.\textsc{def.m.sg} train.\textsc{m.sg} of.\textsc{def.m.sg} train\\
\glt `Be careful when you cross the railway line, the railway line.' (Ruèras, m1, l. 204f.)
\z

The hortative is formed with the first person plural present of the verb \textit{vulaj} `want', \textit{lajn} `let's', and the infinitive (\ref{ex:imp.hort1} and \ref{ex:imp.hort2}).

\ea
\label{ex:imp.hort1}
\gll Ad ùs stù `l bunamajn vagní a métar èls; quaj è tùt, \textbf{lajn} \textbf{dí}, fantasia [...]. \\
and now must.\textsc{prs.3sg} \textsc{3sg.m} really come.\textsc{inf} \textsc{subord} put.\textsc{inf} \textsc{3pl.m} \textsc{dem.unm} \textsc{cop.prs.3sg} all \textsc{imp.1pl} say.\textsc{inf} fantasy.\textsc{f.sg}\\
\glt `And now he must really come and put them [in the right place]; this is all, let’s say, fantasy [...].' (Sèlva, f2, l. 946f.)
\z

\ea
\label{ex:imp.hort2}
\gll \textbf{Lajn} \textbf{còj} quèla tgarn.\\
\textsc{imp.1pl} cook.\textsc{inf} \textsc{dem.f.sg} meat\\
\glt `Let's cook this meat.' (Cavòrgja, f1)
\z

Optative meaning is conveyed by the subjunctive (\ref{ex:opt1}), as the stative verbs do.

\ea
\label{ex:opt1}
\gll Djus \textbf{banadèschi} a \textbf{carschjanti}!\\
God bless.\textsc{prs.sbjv.3sg} and thrive.\textsc{prs.sbjv.3sg}\\
\glt `May God bless [it] and make [it] thrive!' (\DRGoK{5}{649})
\z


\section{Exclamative sentences}
Exclamative sentences are formed with the interrogative pronouns \textit{cù} `how' (\ref{ex:exclcu}) or \textit{tgéj} `what' (\ref{ex:excltgej}).


\ea\label{ex:exclcu}
\gll Jeusas, \textbf{cù} quaj briṣcha!   \\
   \textsc{excl} how \textsc{dem.unm} burn.\textsc{prs.3sg}  \\
\glt `Jee, how it burns!' (\DRGoK{2}{215})
\z

\ea
\label{ex:excltgej}
\gll Míu frá ṣchèva aun ér: «\textbf{Tgé} té pùs!»   \\
\textsc{poss.1sg.m.sg} brother say.\textsc{impf.3sg} still yesterday what \textsc{2sg} can.\textsc{prs.2sg} \\
\glt `My brother said not later than yesterday: «[Incredible that you still] have the strength [to do that]!»' (Sadrún, f3, l. 113)
\z

\section{Voice}

\subsection{Reflexive}

In Tuatschin, reflexive voice is formed with the prefix \textit{sa-} in all finite and non-finite categories. The Sursilvan norm stipulates that the auxiliary verb be \textit{èssar} `be', and in the corpus it is mostly so (\ref{ex:refessar1}--\ref{ex:refessar3}); however, \textit{vaj} `have' is not rare (\ref{ex:refvaj1}--\ref{ex:refvaj3}).\footnote{According to the \DRG{1}{568}, the choice of \textit{esser} as auxiliary verb for reflexives in Sursilvan is due to the demand of Sursilvan grammarians since the 18th century. Nowadays speakers seek to conform to this claim, but in spoken Sursilvan, one still can find \textit{haver} as auxiliary for reflexive verbs, as is the case in Tuatschin.}

\ea\label{ex:refessar1}
\gll [...] ju [...] \textbf{sa-spruava} dad èssar ruassajvals [...].\\
{} \textsc{1sg} {}  \textsc{refl-}try.hard.\textsc{impf.1sg} \textsc{comp} \textsc{cop.inf} calm.\textsc{m.sg}\\
\glt `[...] I [...] tried hard to remain calm [...].' (Sadrún, m10; l. 1070f.)
\z

\ea\label{ex:refessar2}
\gll Lu \textbf{sùnd} ju \textbf{sa-dacidjus} da … raṣdá in pau ṣur da la ... da mi’ ufaunza [...].   \\
then  be.\textsc{prs.1sg}  \textsc{1sg}  \textsc{refl}-decide.\textsc{ptcp.m.sg}  \textsc{comp} {} talk.\textsc{inf} \textsc{indef.m.sg} little over of  \textsc{def.f.sg} {} of \textsc{poss.1sg.f.sg} childhood\\
\glt `Then I decided to ... talk a bit about ... my childhood [...].' (Sadrún, m4; l. 316f.)
\z

\ea\label{ex:refessar3}
\gll  Al tat, scù la mùma ò raquintau, \textbf{è}  \textbf{sa-príuṣ} \textbf{ajn} quaj schi starmantús tga `l è curdauṣ gjùdajn ajn ina rùsna nundétga, ad ò lu stu í a fá cura, mass’ òns, a ... sjantar lu  \textbf{sa-ravagnús} ábar maj pròpi stauṣ ajn gamba.  \\
\textsc{def.m.sg} grandfather as  \textsc{def.f.sg} mother have.\textsc{prs.3sg} tell.\textsc{ptcp.unm} be.\textsc{prs.3sg} \textsc{refl-}take.\textsc{ptcp.m.sg} in \textsc{dem.unm} so terrible.\textsc{adj.unm} \textsc{subord} \textsc{3sg.m} be.\textsc{prs.3sg} fall.\textsc{ptcp.m.sg} down\_into into \textsc{indef.f.sg} hole awful and have.\textsc{prs.3sg} then must.\textsc{ptcp.unm} go.\textsc{inf} \textsc{comp} make.\textsc{inf} treatment.\textsc{f.sg} many year.\textsc{m.pl} and {} after then \textsc{refl}-come\_again.\textsc{ptcp.m.sg} but never really \textsc{cop.ptcp.m.sg} in leg.\textsc{f.sg}\\
\glt `My grandfather, as my mother told [me], took this so seriously that he fell in an awful hole, and for many years he had to go to a health resort, and ... after that he recovered, but he was never really well.' (Sadrún, m4; l. 350ff.)
\z

\ea\label{ex:refvaj1}
\gll  [...] avaun c’ ju sùn staus tial tat savèvu da quaj nuét a \textbf{vèṣ} è bitga \textbf{sa-fatg} \textbf{ajn} zatgé spazjal.\\
{} before \textsc{comp} \textsc{1sg} be.\textsc{prs.1sg} \textsc{cop.ptcp.m.sg} at.\textsc{def.m.sg} grandfather know.\textsc{impf.1sg.1sg} of \textsc{dem.unm} nothing and have.\textsc{cond.1sg} also \textsc{neg} \textsc{refl-}do.\textsc{ptcp.unm} in something special.\textsc{m.sg}\\
\glt `[...] before I stayed with my grandfather I didn’t know anything and I wouldn’t have noticed anything either.' (Sadrún, m4; l. 336ff.)
\z

\ea\label{ex:refvaj2}
\gll Ju a maj gju pròblèm– èl \textbf{vès} maj \textbf{sa-vilau} cun mè né anzatgéj [...].   \\
\textsc{1sg} have.\textsc{prs.1sg} never have.\textsc{ptcp.unm} problem.\textsc{m.sg} \textsc{3sg.m} have.\textsc{cond.3sg} never \textsc{refl-}get\_angry.\textsc{ptcp.unm} with \textsc{1sg} or something\\
\glt `I have never had a problem – he would never have got angry at me or something like that [...].' (Sadrún, m4; l. 614ff.)
\z

\ea\label{ex:refvaj3}
\gll   Als fildiròms \textbf{òn} \textbf{sa-pagljau} ajnt.\\
    \textsc{def.m.pl} wire.\textsc{pl} have.\textsc{prs.3pl} \textsc{refl}-touch.\textsc{ptcp.unm} in \\
\glt `The wires touched each other.' (\DRGoK{6}{321})
\z


\subsection{Reciprocal}
Reciprocal voice is formed with \textit{in ... l'autar} / \textit{ina ... l'autra} `the one ... the other' (\ref{ex:recip1} and \ref{ex:recip2}).

\ea
\label{ex:recip1}
\gll    Ad èlas duaṣ ábar ancanùsché̱van … \textbf{in}’ \textbf{l}’ \textbf{autra} ad ju lu halt bégja.\\
and \textsc{3pl.f} two.\textsc{f.pl} but know.\textsc{impf.3pl} {} one.\textsc{f.sg}  \textsc{def.f.sg} other and \textsc{1sg} then in\_fact \textsc{neg}\\
\glt `But these two already knew … each other but I didn’t.' (Camischùlas, f6, l. 765f.)
\z

\ea
\label{ex:recip2}
\gll Nus vajn dau in cùdisch \textbf{in} \textbf{da} \textbf{l}' \textbf{autar}.\\
\textsc{1pl} have.\textsc{prs.1pl} give.\textsc{ptcp.unm} \textsc{indef.m.sg} book one.\textsc{one.m} \textsc{dat} \textsc{def.m.sg} other\\
\glt `We gave each other a book.' (Sadrún, m5)
\z

\subsection{Causative}
Causative voice is formed with \textit{fá} ‘make’ (\ref{ex:causfa1}--\ref{ex:causfa4}) and \textit{schè/schá}\footnote{\textit{Schá} is the Standard Sursilvan form.} `have something done, let' (\ref{ex:causscha1} and \ref{ex:causscha2}) followed by an infinitive. The causee follows the second verb and occurs as a direct object, which is shown by the use of the direct object pronoun \textit{mè} (vs\textit{ da mé} for the indirect object) in (\ref{ex:causfa5}). Examples (\ref{ex:causfa4}) and (\ref{ex:causfa5}) furthermore show that their construction involves two direct objects.

The analysis of the semantic differences between \textit{fá} and \textit{schè} as causative verbs must be left to further studies.

\ea
\label{ex:causfa1}
\gll    […] i vèvan \textbf{fatg} \textbf{vagní} al caplòn da Sèlva par banadí la nibla […].\\
{}   \textsc{3pl} have.\textsc{impf.3pl} make.\textsc{ptcp.unm} come.\textsc{inf} \textsc{def.m.sg} chaplain of \textsc{pn} \textsc{subord} bless.\textsc{inf} \textsc{def.f.sg} cloud\\
\glt `They’d had the chaplain of Selva come in order to bless the cloud […].' (Ruèras, \citealt[62]{Büchli1966})
\z

\ea
\label{ex:causfa2}
\gll     Als da Sadrún òn vulju \textbf{fá} \textbf{stá} \textbf{anavùs} la buéba […].\\
\textsc{def.m.pl} of Sedrun have.\textsc{prs.3pl} want.\textsc{ptcp.unm} make.\textsc{inf} stay.\textsc{inf} back \textsc{def.f.sg} girl\\
\glt `The people of Sedrun wanted to have the girl remain there.' (Bugnaj, \citealt[131]{Büchli1966})
\z


\ea\label{ex:causfa3}
\gll   [...] plaunsjú ṣèni vagní da \textbf{fá} \textbf{í} scha vèva `l rùt in calum.\\
{} slowly be.\textsc{impf.3pl} come.\textsc{ptcp.m.pl} \textsc{comp} make.\textsc{inf} go.\textsc{inf} since have.\textsc{impf.3sg} \textsc{3sg.m} break.\textsc{ptcp.unm} \textsc{indef.m.sg} thigh \\
\glt `[...] they succeeded slowly in having [him] go [to the hospital] since he had broken a thigh.' (Sadrún, m4, l. 636f.)
\z

\ea
\label{ex:causfa4}
\gll Ju \textbf{fétsch} \textbf{fá} al cusunz in pèr tgautschas pr̩ èl.\\
\textsc{1sg} make.\textsc{prs.1sg} do.\textsc{inf} \textsc{def.m.sg} tailor \textsc{indef.m.sg} pair trousers.\textsc{m.pl} for \textsc{3sg.m}\\
\glt `I have the tailor make a pair of trousers for him.' (Sadrún, m4)
\z

\ea
\label{ex:causfa5}
\gll Èl \textbf{fò} \textbf{fá} \textbf{mè} in pèr tgautschas pr̩ èl.\\
\textsc{3sg.m} make.\textsc{prs.3sg} do.\textsc{inf} \textsc{1sg.do} \textsc{indef.m.sg} pair trousers.\textsc{m.pl} for \textsc{3sg.m}\\
\glt `He has me make a pair of trousers for him.' (Sadrún, m5 and m6)
\z

\ea
\label{ex:causscha1}
\gll I \textbf{schèvan} \textbf{luvrá} fétg.\\
\textsc{3pl} let.\textsc{impf.3pl} work.\textsc{inf} much\\
\glt `They had [us] work a lot.' (Ruèras, f4, 1. 1940)
\z

\ea
\label{ex:causscha2}
\gll    Èl ò \textbf{schau} \textbf{savaj} la règína quaj.\\
\textsc{3sg.m} have.\textsc{prs.3sg} let.\textsc{ptcp.unm} know.\textsc{inf} \textsc{def.f.sg} queen \textsc{dem}\\
\glt `He let the queen know this.' (Sadrún, m5)
\z

Standard Sursilvan possesses the derivational suffix -\textit{entar} which transforms a verb or another syntactic category into a causative. Tuatschin also possesses this suffix, -\textit{antá} in the spelling used in this grammar, but to a very reduced extent. Where Standard Sursilvan has \textit{cuschentar} `cause to be quiet', \textit{fughentar} `light a fire', or \textit{luchentar} (\textit{il tratsch}) `loosen (the soil)', Tuatschin has \textit{fá còschar}, \textit{dá fjuc}, and \textit{fá luc} (\textit{al tratsch}). The causative verbs with -\textit{antá} which occur in the corpus are presented in \tabref{factanta}. These verbs can be derived from verbs, adjectives, or nouns.

\begin{table}
	\caption{Factitive verbs}
	\label{factanta}
	\begin{tabular}{lllll}
		\lsptoprule
		\textit{bubrantá} &`make drunk' & < & \textit{bájbar} & `drink'\\
		\textit{buantá} & `water (animal)' & < & \textit{bájbar} & `drink'\\
		\textit{cuntantá} &`satisfy' & < & \textit{cuntjants} & `glad'\\
		\textit{durmantá} & `make sleep' & < & \textit{durmí} & `sleep'\\
		\textit{fimantá} &`smoke' & < & \textit{fém} & `smoke (n.)'\\
		\textit{luantá}	&`melt (tr.)' & < &\textit{luá} & `melt (itr.)'\\
		\textit{nagantá}	&`drown (tr.)' & < & \textit{nagá} & `drown (itr.)'\\
		\textit{schjantá} & `dry' & < & \textit{schétg} & `dry (adj.)'\\
		\textit{schlupantá} & `blow up'&	< & \textit{schlupá} & `explode'\\
		\textit{sagljantá} & `blow up' & < & \textit{saglí} & `run, jump'\\
		
		\lspbottomrule
	\end{tabular}
\end{table}

The causative verbs show a stem alternation like in \textit{ju bubrjanta} `I make drunk' (vs \textit{nuṣ bubrantajn} `we make drunk'). The verbs \textit{schjantá} `dry' and \textit{sagljantá} `blow' do not exhibit this alternation because they already have the diphthong \textit{ja} in their stem.

\subsection{Passive}
A dynamic passive is formed with the verb\textit{vagní} ‘come’ and the participle (\ref{ex:pass.dyn1}--\ref{ex:pass.dyn5}). A stative passive is formed with the verb \textit{èssar} `be' and the participle (\ref{ex:pass.stat1}--\ref{ex:pass.stat3}). In both cases the participle agrees with the patient subject if it precedes the passive construction.

\ea
\label{ex:pass.dyn1}
\gll    [...] \textbf{als} \textbf{tiars} èn \textbf{vagní} \textbf{pri} òd stával a \textbf{purtaj} navèn.\\
{}    \textsc{def.m.pl}  animal.\textsc{pl}  be.\textsc{prs.3pl}   \textsc{pass.ptcp.m.pl}   take.\textsc{ptcp.m.pl}  out\_of barn and bring.\textsc{ptcp.m.pl}  away\\
\glt `The animals were taken out of the barn and brought away.' (Tschamùt, \citealt[53]{Büchli1966})
\z

\ea
\label{ex:pass.dyn2}
\gll \textbf{Èla} ségi \textbf{vagnida} \textbf{tratga} cun starmantusa fòrza […].\\
\textsc{3sg.f} \textsc{cop.prs.sbjv.3sg} come.\textsc{pass.ptcp.f.sg} pull.\textsc{ptcp.f.sg} with tremendous.\textsc{f.sg} power\\
\glt `[She said that] she had been pulled with tremendous power.' (Bugnaj, \citealt[132]{Büchli1966})
\z

\ea
\label{ex:pass.dyn3}
\gll    [...] \textbf{nus} \textbf{vagnévan} pròpi \textbf{tanidas} a nus stèvan amprèndar a nus stèvan ṣchubargè a fá a tùt.\\
{} \textsc{1pl} \textsc{pass.impf.3pl} really hold.\textsc{ptcp.f.pl} and \textsc{1pl} must.\textsc{impf.1pl} learn.\textsc{inf} and \textsc{1pl}  must.\textsc{impf.1pl} clean.\textsc{inf} and do.\textsc{inf} and all\\
\glt `[...] we were really kept [in a strict way] and we had to study and we had to clean and do and everything.' (Camischùlas, f6; l. 677ff.)
\z

If the patient subject follows the passive verb, there is no agreement.

\ea
\label{ex:pass.dyn4}
\gll    [...] a lò végn\textbf{i} fatg \textbf{mèssa}, sunau \textbf{òrgla} a cantau \textbf{végljas} \textbf{canzu̱ns} \textbf{ròmò̱ntschas} […].\\
{} and there \textsc{pass.prs.3sg}.\textsc{expl} do.\textsc{ptcp.unm} mass.\textsc{f.sg} play.\textsc{ptcp.m.sg} organ.\textsc{f.sg}  and sing.\textsc{ptcp.m.sg} old.\textsc{f.pl} song.\textsc{pl} Romansh.\textsc{pl}\\
\glt `[…] and there a mass is said, the organ is played, and old Romansh songs are sung […].' (Camischùlas, \citealt[94]{Büchli1966})
\z

\ea
\label{ex:pass.dyn5}
\gll    [...] lu èra quaj … craju, sjat fjastas … tga \textbf{vagnéva} … \textbf{fátg} \textbf{parada}.\\
{} then \textsc{cop.impf.3sg} \textsc{dem.unm} {} believe.\textsc{prs.1sg.1sg} seven celebration.\textsc{f.pl} {} \textsc{rel} \textsc{pass.impf.3sg} {} do.\textsc{ptcp.unm} parade.\textsc{f.sg}\\
\glt `[...] then there were …, I believe, seven celebrations … when they would … prepare a parade.' (Zarcúns, m2; l. 1543ff.)
\z

\ea
\label{ex:pass.stat1}
\gll A qu' \textbf{èra} schòn \textbf{dau} bjè \textbf{najv} ad èran bigj’ aun vagní vidò culs tiars.\\  
and \textsc{dem.unm} \textsc{pass.impf.3sg} already give.\textsc{ptcp.unm} much snow.\textsc{f.sg} and be.\textsc{impf.3pl} \textsc{neg} yet come.\textsc{ptcp.m.pl} down with.\textsc{def.m.pl} animal.\textsc{pl} \\
\glt `And there was already a lot of snow and they hadn’t come back down with the animals yet.' (Sadrún, m4; l. 592f.)
\z

\ea
\label{ex:pass.stat2}
\gll Ad ùssa òni partju ajn quaj, al cantún ò circa trènta da quèls majnadistri̱cts, \textbf{quèls} \textbf{èn} \textbf{partí} \textbf{ajn} ajn ragjúns, ad ju a la val Tujétsch a Musté.\\
and now have.\textsc{prs.3pl.3pl} divide.\textsc{ptcp.unm} in \textsc{dem.unm} \textsc{def.m.sg} canton  have.\textsc{prs.3sg} about thirty of \textsc{dem.m.pl} head\_of\_district.\textsc{pl} \textsc{dem.m.pl} \textsc{pass.prs.3pl} divide.\textsc{ptcp.m.pl} in in region.\textsc{f.pl} and \textsc{1sg} have.\textsc{prs.1sg} \textsc{def.f.sg} valley \textsc{pn} and \textsc{pn}\\
\glt `And now they have divided that, the canton has about thirty of these heads of district, these are divided into regions, and I have the Tujetsch valley and Mustér.' (Sadrún, f3; l. 106ff.)
\z

\ea
\label{ex:pass.stat3}
\gll Ins vèz’ aun tg’ \textbf{èra} \textbf{dau} vidajn \textbf{pùntgas} né \textbf{trádals} [...].   \\
\textsc{gnr} see.\textsc{prs.3sg} still \textsc{comp} \textsc{pass.impf.3sg} give.\textsc{ptcp.unm} into chisel.\textsc{f.pl} or power\_drill.\textsc{m.pl}\\
\glt `One still can see that chisels or power drills had been used [...].' (Sadrún, m4; l. 437f.)
\z

Place names are considered to have no gender, hence the use of the unmarked form of the participle (\ref{ex:pass.dyn6}).

\ea
\label{ex:pass.dyn6}
\gll Ah… Nalps \textbf{è} \textbf{vagnú} \textbf{fraquantau} \textbf{ò} scù majṣès ad alps adina [...].\\
ah  \textsc{pn}  be.\textsc{prs.3sg} \textsc{pass.ptcp.unm} visit.\textsc{ptcp.unm} out as assembly\_of\_houses and alp.\textsc{m.pl} always\\
\glt `Eh … Nalps has always been visited as an assembly of houses and as pastures [...].' (Sadrún, m4; l. 409f.)
\z

If the agent of a passive construction is mentioned, it is introduced by \textit{da} (\ref{ex:pass.dyn7} and \ref{ex:pass.dyn8}). Whether \textit{da} corresponds to the preposition or to the dative marker is not easy to decide. The only case where there is a difference between the two \textit{da}'s is the first person object pronoun, which is either \textit{mè} (accusative and after prepositions) or \textit{mé} (dative marker). As noted above in §3.5, some speakers prefer using the pronoun \textit{mé} (dative), whereas others use \textit{mè} (accusative) in order to introduce the agent of a passive construction.

\ea
\label{ex:pass.dyn7}
\gll [...] quèls mulissiars, quèls ah fagèvan lu quasi las préfatschèntas, né, né tg’ èran ... cumissunaj \textbf{da} \textbf{quèls} ah \textbf{martgadònts} grònṣ da la bassa, né.\\
{} \textsc{dem.m.pl} negotiator.\textsc{pl} \textsc{dem.m.pl} eh do.\textsc{impf.3pl} then so\_to\_speak \textsc{def.f.pl}  intermediate\_trade.\textsc{pl} right or \textsc{comp} \textsc{pass.impf.3pl} {} commission.\textsc{ptcp.m.pl} \textsc{dat} \textsc{dem.m.pl} eh businessman.\textsc{pl} big.\textsc{pl} of \textsc{def.f.sg} «lowlands» right \\
\glt `[...] These negotiators, they would so to speak do the intermediate trade, or they were ... commissioned by the big businessmen from outside the Grisons, right?' (Sadrún, m5; l. 1219ff.)
\z

\ea
\label{ex:pass.dyn8}
\gll Al baghètg è vagnuṣ bagagjauṣ \textbf{da} \textbf{mju} \textbf{auc}.\\
\textsc{def.m.sg} building be.\textsc{prs.3sg} \textsc{pass.ptcp.m.sg} build.\textsc{ptcp.m.sg} \textsc{dat} \textsc{poss.1sg.m.sg} uncle\\
\glt `The building has been built by my uncle.' (Sadrún, m1)
\z
